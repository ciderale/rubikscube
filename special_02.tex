\tikzset{front/.style={fill=red}}
\tikzset{top/.style={fill=green}}
\tikzset{right/.style={fill=white}}
\tikzset{left/.style={fill=yellow}}
\tikzset{back/.style={fill=orange}}
\tikzset{bottom/.style={fill=blue}}

\begin{tikzpicture}
  \tikzset{arrowlabel/.style={pos=0.75,draw=black,fill=white,circle,inner sep=0.5pt,font=\sffamily\bfseries}}
  \tikzset{ultrathick/.style={line width=2}}

  \begin{scope}
  [x={(0.9cm,-0.1cm)},y={(0,0.9cm)},z={(0.6cm,0.4cm)},scale=1.1]
% BACK
\drawCubicleNorthEast[X]{-1}{-1}{2}
\drawCubicleNorthEast[X]{-1}{-0}{2}
\drawCubicleNorthEast[X]{-0}{-1}{2}
\drawCubicleNorthEast[X]{-1}{+1}{2}
\drawCubicleNorthEast[X]{-0}{+0}{2}
\drawCubicleNorthEast[X]{+1}{-1}{2}
\drawCubicleNorthEast[X]{-0}{+1}{2}
\drawCubicleNorthEast[X]{+1}{-0}{2}
\drawCubicleNorthEast[X]{+1}{+1}{2}
%MIDDLE
\drawCubicleNorthEast[X]{-1}{-1}{1}
\drawCubicleNorthEast[X]{-1}{-0}{1}
\drawCubicleNorthEast[X]{-0}{-1}{1}
\drawCubicleNorthEast[X]{-1}{+1}{1}
\drawCubicleNorthEast[X]{-0}{+0}{1}
\drawCubicleNorthEast[X]{+1}{-1}{1}
\drawCubicleNorthEast[X]{-0}{+1}{1}
\drawCubicleNorthEast[X]{+1}{-0}{1}
\drawCubicleNorthEast[X]{+1}{+1}{1}
% FRONT
\drawCubicleNorthEast[X]{-1}{-1}{0}
\drawCubicleNorthEast[X]{-1}{-0}{0}
\drawCubicleNorthEast[X]{-0}{-1}{0}
\drawCubicleNorthEast[X]{-1}{+1}{0}
\drawCubicleNorthEast[X]{-0}{+0}{0}
\drawCubicleNorthEast[X]{+1}{-1}{0}
\drawCubicleNorthEast[X]{-0}{+1}{0}
\drawCubicleNorthEast[X]{+1}{-0}{0}
\drawCubicleNorthEast[X]{+1}{+1}{0}
%% FRONT
  \begin{scope}[canvas is xz plane at y=2.0]
    \draw[fill=black, fill opacity=0.5,even odd rule]
      (0.5,1.5) circle (1.4)
      (0.5,1.5) circle (1.0)
    ;
    \draw (0.5,1.5) node[text=white,fill=black] {\bf turn};
  \end{scope}
  \begin{scope}[canvas is yz plane at x=2.0]
    \draw[fill=black, fill opacity=0.5,even odd rule]
      (0.5,1.5) circle (1.4)
      (0.5,1.5) circle (1.0)
    ;
    \draw (0.5,1.5) node[text=white,fill=black] {\bf turn};
  \end{scope}
  \begin{scope}[canvas is xy plane at z=0]
    \fill[pattern=checkerboard,opacity=0.5] (-1,-1) rectangle ++(2,2) node[midway,opacity=1,text=white,fill=black]{\bf hold};
    \draw[->,ultrathick] (-0.5,+1.5+0.2) -- ++(2.0,0) node[arrowlabel]{R};
    \draw[->,ultrathick] (+1.5+0.2,-0.5) -- ++(0,2.0) node[arrowlabel]{U};
    \draw[->,ultrathick] (-0.5+2,+1.5-0.2) -- ++(-2.0,0) node[arrowlabel]{L};
    \draw[->,ultrathick] (+1.5-0.2,-0.5+2) -- ++(0,-2.0) node[arrowlabel]{D};
  \end{scope}
\end{scope}


  \draw (1.5,0.5) coordinate(thecenter);
  \def\theradius{6}
\begin{pgfonlayer}{background}
  \draw[fill=black, fill opacity=0.5,even odd rule]
  (thecenter) circle (\theradius+0.25)
  (thecenter) circle (\theradius-0.25);
\end{pgfonlayer}

  % define cube face color in "neutral" position
  \def\myarray{{{"red","yellow","green"},{"red","white","blue"}}}

  \foreach \i/\opa in {0/1, 1/1, 2/0.6, 3/0.3, 4/0.6, 5/1} {
    \path[->] (thecenter) -- ++(-\i*60:\theradius) coordinate(A);


    \begin{scope}[shift={(A)},fill opacity=1,local bounding box=rot\i]%\opa]
      \fill[pale,fill opacity=0.3] (-1,-1) rectangle ++(3,3);
      %\fill[red] (-1,-1) rectangle ++(2,2);
      \fill[pattern=checkerboard,opacity=0.3] (-1,-1) rectangle ++(2,2);
      \draw (-1,-1) grid ++(3,3);
      \draw (-1,-1) rectangle ++(3,3);

      \pgfmathsetmacro{\nodename}{\i}
      \draw (0.5,0.5) node[fill=black,text=white]{\bf \nodename};

      \begin{scope}[z={(-0.5cm,+0.5cm)}]
        % Lookup of color rotation
        \pgfmathsetmacro{\currentcolourA}{\myarray[Mod(\i,2)][Mod(\i+0,3)]}
        \tikzset{front/.style={fill=\currentcolourA}}
        \pgfmathsetmacro{\currentcolourB}{\myarray[Mod(\i,2)][Mod(\i+1,3)]}
        \tikzset{left/.style={fill=\currentcolourB}}
        \pgfmathsetmacro{\currentcolourC}{\myarray[Mod(\i,2)][Mod(\i+2,3)]}
        \tikzset{top/.style={fill=\currentcolourC}}
        % draw the cube
        \drawCubicleNorthWest[X]{-1}{+1}{0}
      \end{scope}
      \begin{scope}[z={(+0.5cm,-0.5cm)}]
        % Lookup of color rotation
        \pgfmathsetmacro{\currentcolourA}{\myarray[Mod(\i+1,2)][Mod(\i+0,3)]}
        \tikzset{front/.style={fill=\currentcolourA}}
        \pgfmathsetmacro{\currentcolourB}{\myarray[Mod(\i+1,2)][Mod(\i+1,3)]}
        \tikzset{right/.style={fill=\currentcolourB}}
        \pgfmathsetmacro{\currentcolourC}{\myarray[Mod(\i+1,2)][Mod(\i+2,3)]}
        \tikzset{bottom/.style={fill=\currentcolourC}}
        % draw the cube
        \drawCubicleSouthEast[X]{+1}{-1}{0}
      \end{scope}
    \end{scope}

    \ifthenelse{\equal{\i}{0}}{
      \begin{scope}[shift={(A)},z={(-0.5cm,+0.5cm)},canvas is yz plane at x=-1]
        \draw[very thick,pattern=crosshatch] (1,0) rectangle ++(1,1);
      \end{scope}
    }{}
  }

\begin{scope}[shift={(+\theradius+1.5,+4.5)},local bounding box=BB]
  \draw (0,0) coordinate(X);
  \draw[>->,ultrathick] (X) -- ++(2,0) node[arrowlabel]{R} coordinate(X);
  \draw[->,ultrathick] (X) -- ++(0,2) node[arrowlabel]{U} coordinate(X);
  \draw[->,ultrathick] (X) -- ++(-2,0) node[arrowlabel]{L} coordinate(X);
  \draw[->,ultrathick] (X) -- ++(0,-1.8) node[arrowlabel]{D} coordinate(X);
  \draw (1,1) node{\bf +1};
  \begin{pgfonlayer}{background}
    \fill[magenta,opacity=0.7] (BB.north west) rectangle (BB.south east);
  \end{pgfonlayer}
\end{scope}

\begin{scope}[shift={(+\theradius+1.5,-4.5)},local bounding box=BB]
  \draw (0,0) coordinate(X);
  \draw[>->,ultrathick] (X) -- ++(0,2) node[arrowlabel]{U} coordinate(X);
  \draw[->,ultrathick] (X) -- ++(2,0) node[arrowlabel]{R} coordinate(X);
  \draw[->,ultrathick] (X) -- ++(0,-2) node[arrowlabel]{D} coordinate(X);
  \draw[->,ultrathick] (X) -- ++(-1.8,0) node[arrowlabel]{L} coordinate(X);
  \draw (1,1) node{\bf -1};
  \begin{pgfonlayer}{background}
    \fill[orange,opacity=1.0] (BB.north west) rectangle (BB.south east);
  \end{pgfonlayer}
\end{scope}

{
\tikzset{shortarrow/.style={shorten >=4, shorten <=4}}
\tikzset{tiplabel/.style={font=\sffamily\bfseries,anchor=east,xshift=-5}}
\tikzset{bgarrow/.style={line width=6,white}}
\tikzset{fgarrow/.style={line width=2.5,magenta}}

\draw[ultrathick,shortarrow,bgarrow]
  ($(rot5.south east) + (-1,0)$)
  to[out=-130,in=130,looseness=1]
  (rot0.north west);
\draw[ultrathick,shortarrow,fgarrow,->]
  ($(rot5.south east) + (-1,0)$)
  to[out=-130,in=130,looseness=1]
  (rot0.north west)
  node[tiplabel] {+1};
\draw[ultrathick,shortarrow,bgarrow]
  ($(rot1.south east) + (-1.0,1.5)$)
  to[out=100,in=-110,looseness=1]
  ($(rot0.north west) + (+0.4,-1.5)$);
\draw[ultrathick,shortarrow,fgarrow,->, orange] %override for -1
  ($(rot1.south east) + (-1.0,1.5)$)
  to[out=100,in=-110,looseness=1]
  ($(rot0.north west) + (+0.4,-1.5)$)
  node[tiplabel,yshift=-5]{-1};
\draw[ultrathick,shortarrow,bgarrow]
  ($(rot3.south east) + (0,0.5)$)
  to[out=-30,in=-130,looseness=1.4]
  ($ (rot0.north west) + (0,-1) $);
\draw[ultrathick,shortarrow,fgarrow,->,dotted]
  ($(rot3.south east) + (0,0.5)$)
  to[out=-30,in=-130,looseness=1.4]
  ($ (rot0.north west) + (0,-1) $)
  node[tiplabel]{$\pm$3};

\draw[ultrathick,shortarrow,bgarrow]
  ($(rot2.north west) + (0,-0.5)$) coordinate (Rot2Start) --
  ++(-2.2,0) |-
  ($ (rot4.north west) + (0,-1) $) coordinate (Rot2End);
\draw[ultrathick,shortarrow,fgarrow,->]
  ($(rot2.north west) + (0,-0.5)$) --
  ++(-2.2,0) |-
  ($ (rot4.north west) + (0,-1) $)
  node[tiplabel,yshift=-10]{+2};
}

{
  %TODO: maybe make that visually too
%\tikzset{default/.style={anchor=north east,text width=2.5cm,yshift=-0.1cm,font=\sffamily\bfseries}}
%\draw (Rot2Start) node[default]{counter-clock};
%\draw (Rot2End) node[default,yshift=-1cm]{clockwise};

\begin{scope}[
     shift={(-\theradius+1.5,+4.5)},local bounding box=BB, %top
     %shift={(-\theradius+1.5,-3.45)},local bounding box=BB, %bottom
     z={(-0.5cm,+0.5cm)},
     scale=0.8
]
  % Lookup of color rotation
  \pgfmathsetmacro{\i}{2} %top
  %\pgfmathsetmacro{\i}{4} %bottom
  \pgfmathsetmacro{\currentcolourA}{\myarray[Mod(\i,2)][Mod(\i+0,3)]}
  \tikzset{front/.style={fill=\currentcolourA}}
  \pgfmathsetmacro{\currentcolourB}{\myarray[Mod(\i,2)][Mod(\i+1,3)]}
  \tikzset{left/.style={fill=\currentcolourB}}
  \pgfmathsetmacro{\currentcolourC}{\myarray[Mod(\i,2)][Mod(\i+2,3)]}
  \tikzset{top/.style={fill=\currentcolourC}}
  % draw the cube
  \drawCubicleNorthWest[X]{-1}{+1}{0}

  \draw[ultrathick,->] (-0.5,1.5) arc[start angle=-45, end angle=-165, radius=0.45];
  \draw[ultrathick,->] (-0.5,1.5) arc[start angle=-45, end angle=-300, radius=0.45];
  \draw[ultrathick,->] (-0.5,1.5) arc[start angle=-45, end angle=-405, radius=0.45];
  \draw (-0.8,0.8) node[font=\sffamily\bfseries]{clockwise};
\end{scope}


\begin{scope}[shift={(-\theradius+0.5,-5.5)},local bounding box=BB]
  \draw (0,0) coordinate(X);
  \draw[>->,ultrathick] (X) -- ++(2,0) node[arrowlabel]{R} coordinate(X);
  \draw[->,ultrathick] (X) -- ++(0,2) node[arrowlabel]{U} coordinate(X);
  \draw[->,ultrathick] (X) -- ++(-2,0) node[arrowlabel]{L} coordinate(X);
  \draw[->,ultrathick] (X) -- ++(0,-1.6) node[arrowlabel]{D} coordinate(X);
  %
  \draw[->,ultrathick] (X) -- ++(1.6,0)  coordinate(X);
  \draw[->,ultrathick] (X) -- ++(0,1.2)  coordinate(X);
  \draw[->,ultrathick] (X) -- ++(-1.2,0) coordinate(X);
  \draw[->,ultrathick] (X) -- ++(0,-0.8) coordinate(X);
  \draw (1,1) node{\bf +2};
  \begin{pgfonlayer}{background}
    \fill[magenta,opacity=0.7] (BB.north west) rectangle (BB.south east);
  \end{pgfonlayer}

  \draw[ultrathick,->,white] (0.7,0.7) arc[start angle=-135, end angle=+135, radius=0.40];
  \draw (1,-0.4) node[font=\sffamily\bfseries]{counter-clockwise};
\end{scope}
}


\end{tikzpicture}%
